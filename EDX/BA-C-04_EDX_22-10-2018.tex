% Autor: Leonhard Segger, Alexander Neuwirth
% Datum: 2017-10-30
\documentclass[
	% Papierformat
	a4paper,
	% Schriftgröße (beliebige Größen mit „fontsize=Xpt“)
	12pt,
	% Schreibt die Papiergröße korrekt ins Ausgabedokument
	pagesize,
	% Sprache für z.B. Babel
	ngerman
]{scrartcl}

% Achtung: Die Reihenfolge der Pakete kann (leider) wichtig sein!
% Insbesondere sollten (so wie hier) babel, fontenc und inputenc (in dieser
% Reihenfolge) als Erstes und hyperref und cleveref (Reihenfolge auch hier
% beachten) als Letztes geladen werden!

\usepackage{tikz}
\usetikzlibrary{calc,patterns,angles,quotes} % loads some tikz extensions\usepackage{tikz}
\usetikzlibrary{babel}

% Silbentrennung etc.; Sprache wird durch Option bei \documentclass festgelegt
\usepackage{babel}
% Verwendung der Zeichentabelle T1 (Sonderzeichen etc.)
\usepackage[T1]{fontenc}
% Legt die Zeichenkodierung der Eingabedatei fest, z.B. UTF-8
\usepackage[utf8]{inputenc}
% Schriftart
\usepackage{lmodern}
% Zusätzliche Sonderzeichen
\usepackage{textcomp}

% Mathepaket (intlimits: Grenzen über/unter Integralzeichen)
\usepackage[intlimits]{amsmath}
% Ermöglicht die Nutzung von \SI{Zahl}{Einheit} u.a.
\usepackage{siunitx}
% Zum flexiblen Einbinden von Grafiken (\includegraphics)
\usepackage{graphicx}
% Abbildungen im Fließtext
\usepackage{wrapfig}
% Abbildungen nebeneinander (subfigure, subtable)
\usepackage{subcaption}
% Funktionen für Anführungszeichen
\usepackage{csquotes}
\MakeOuterQuote{"}
% Zitieren, Bibliografie
\usepackage[sorting=none]{biblatex}
\usepackage{csvsimple}


% Zur Darstellung von Webadressen
\usepackage{url}
%chemische Formeln
\usepackage[version=4]{mhchem}
% siunitx: Deutsche Ausgabe, Messfehler getrennt mit ± ausgeben
\usepackage{floatrow}
\floatsetup[table]{capposition=top}
\usepackage{float}
% Verlinkt Textstellen im PDF-Dokument
\usepackage[unicode]{hyperref}
% "Schlaue" Referenzen (nach hyperref laden!)
\usepackage{cleveref}
\sisetup{
	locale=DE,
	separate-uncertainty
}
%\bibliography{BA-C-04_EDX_22-10-2018_References}


\begin{document}
	
	\begin{titlepage}
		\centering
		{\scshape\LARGE Versuchsbericht zu \par}
		\vspace{1cm}
		{\scshape\huge EDX - Energiedispersive Röntgenspektroskopie \par}
		\vspace{2.5cm}
		{\LARGE Gruppe BA-C-04 \par}
		\vspace{0.5cm}
		
		{\large Alexander Neuwirth (E-Mail: a\_neuw01@wwu.de) \par}
		{\large Leonhard Segger (E-Mail: l\_segg03@uni-muenster.de) \par}
		\vfill
		
		durchgeführt am 22.10.2018\par
		betreut von\par
		{\large Johann Preuß} %TODO Johann Adrian Preuß? Irrelephant?!?
		
		\vfill
		
		{\large \today\par}
	\end{titlepage}
	\tableofcontents
	\newpage

	%TODO mehr TODO in Default	

	\section{Kurzfassung}
	%TODO Hypothese	und deren Ergebnis, wenn Hypothese ist, dass nur Theorie erfüllt, sagen: Erwartung: Theorie aus einführung (mit reflink) erfüllt
	%TODO Ergebnisse, auch Zahlen, mindestens wenn's halbwegs Sinn ergibt
	%TODO Was wurde gemacht
	%TODO manche leute wollen Passiv oder "man", manche nicht
	
	%\section{Einführung} %optional
	
	
	\section{Methoden}
	%TODO Bilder von der Website klauen
	%TODO einer will Präsens
	
	%TODO erwähnen max energie 35kev strahlugn
	%TODO so klaibrier gelümps
	
	\section{Ergebnisse und Diskussion}
	%TODO Unsicherheiten
	

	\subsection{Beobachtung}
	%TODO Einflüsse von veränderten Parametern auf Messung
	\subsubsection{Unsicherheiten} %TODO GGF IN DATENANYLSY
	\subsection{Datenanalyse}
	%TODO Berechung nach Aufgabenstellung
	Aus den gemessenen Energiespektren wurden die Energien der Peaks mittels eines Gauß-Fit bestimmt.
	Die Standardabweichung ergibt sich dabei aus der FWHM:
	\begin{equation}
		\sigma = \frac{\text{FWHM}}{2\sqrt{\ln 2}}
	\end{equation}
	Die Ergebnisse sind in \cref{tb_peaks_known} und \cref{tb_peaks_unknown} aufgeführt.
	
	%TODO sortieren nach enrgie oder Ausprägung
	
	\begin{table}[H]
		\centering
		\begin{tabular}{ c | c || c | c }
			Probe (Angabe)&Energie $E$ in \SI{}{keV} & Element (char. Übergang) &  Energie $E$ in \SI{}{keV} \\ \hline \hline
			
			1 (Zn)& \SI{8.58804+-0.233242}{} &Zn (K$\alpha_2$)&  \SI{8.61622(50)}{} \\
			& \SI{9.53296+-0.221475}{} &Zn (K$\beta_1$) &  \SI{9.5227(14)}{} \\
			& \SI{6.38034+-0.278836}{} & - &  - \\ \hline
			
			2 (Fe)& \SI{6.39075+-0.235386}{} &Fe (K$\alpha$) &  \SI{6.4030}{} \\
			& \SI{7.08998+-0.190662}{} &Fe &  \SI{7.0570}{} \\
			& \SI{3.44288+-0.335591}{} &- &  - \\ \hline
			
			3 (Cu)& \SI{8.00053+-0.233583}{} &Cu (K$\alpha_2$) &  \SI{8.04811(45)}{} \\
			& \SI{8.87574+-0.209472}{} &Cu (K$\beta_2$) &  \SI{8.9040(11) }{}  \\ \hline
			
			4 (20 Cent)& \SI{7.98262+-0.224069}{} &Zn &  \SI{}{} \\
			& \SI{8.7734+-0.247708}{} &Zn &  \SI{}{} \\ \hline
			
			5 (Zn-Edelstahl)& \SI{8.5778+-0.252635}{} &Zn &  \SI{}{} \\
			& \SI{9.53652+-0.22887}{} &Zn &  \SI{}{} \\
			& \SI{6.41683+-0.35613}{} &Zn &  \SI{}{} \\ \hline
			
			6 (Edelstahl)& \SI{6.36975+-0.246009}{} &Zn &  \SI{}{} \\
			& \SI{7.07128+-0.203035}{} &Zn &  \SI{}{} \\
			& \SI{3.39677+-0.379146}{} &Zn &  \SI{}{} \\ \hline
			
			7 (Ti)& \SI{4.54007+-0.270266}{} &Zn &  \SI{}{} \\
			& \SI{6.40999+-0.297794}{} &Zn &  \SI{}{} \\
			& \SI{2.54705+-0.562162}{} &Zn &  \SI{}{} \\
			& \SI{1.39218+-0.0951656}{} &Zn &  \SI{}{} \\ \hline
			
			8 (Mo)& \SI{17.3891+-0.263783}{} &Zn &  \SI{}{} \\
			& \SI{19.5816+-0.291623}{} &Zn &  \SI{}{} \\
			& \SI{7.79212+-0.572622}{} &Zn &  \SI{}{} \\
			& \SI{11.1937+-0.586733}{} &Zn &  \SI{}{} \\ \hline
			
		\end{tabular}
		\caption{Gemessene Röntgenfluoreszenzmaxima. Die Vergleichsenergien wurden dem Periodensystem des Programms Phywe Measure 4 entnommen.}
		\label{tb_peaks_known} 
	\end{table}
	\begin{table}[H]
		\centering
		\begin{tabular}{ c | c || c | c }
			Probe (Angabe)&Energie $E$ in \SI{}{keV} & vermt. Element (char. Übergang) &  Energie $E$ in \SI{}{keV} \\ \hline \hline		
			9 & \SI{6.36714+-0.297413}{} &Zn &  \SI{}{} \\
			& \SI{9.52768+-0.218296}{} &Zn &  \SI{}{} \\
			& \SI{8.57041+-0.24602}{} &Zn &  \SI{}{} \\ \hline
			
			10 & \SI{7.4251+-0.243006}{} &Zn &  \SI{}{} \\
			& \SI{8.23717+-0.212733}{} &Zn &  \SI{}{} \\ \hline
			
			11 & \SI{7.79736+-0.404885}{} &Zn &  \SI{}{} \\
			& \SI{9.00316+-0.164444}{} &Zn &  \SI{}{} \\
			& \SI{5.92928+-0.455814}{} &Zn &  \SI{}{} \\ \hline
			
			12 & \SI{10.5011+-0.261044}{} &Zn &  \SI{}{} \\
			& \SI{12.562+-0.289138}{} &Zn &  \SI{}{} \\
			& \SI{14.7821+-0.364269}{} &Zn &  \SI{}{} \\
			& \SI{9.11464+-0.376278}{} &Zn &  \SI{}{} \\ \hline
			
			13 & \SI{8.00052+-0.25263}{} &Zn &  \SI{}{} \\
			& \SI{14.8774+-0.26906}{} &Zn &  \SI{}{} \\
			& \SI{4.7454+-0.480512}{} &Zn &  \SI{}{} \\
			& \SI{8.88531+-0.246938}{} &Zn &  \SI{}{} \\
			& \SI{16.7011+-0.297662}{} &Zn &  \SI{}{} \\ \hline
			
			14 & \SI{3.07729+-0.327119}{} &Zn &  \SI{}{} \\
			& \SI{7.7277+-0.526665}{} &Zn &  \SI{}{} \\
			& \SI{11.8938+-0.476134}{} &Zn &  \SI{}{} \\
			& \SI{15.0918+-0.63684}{} &Zn &  \SI{}{} \\
			& \SI{18.1496+-0.886139}{} &Zn &  \SI{}{} \\
			& \SI{22.0709+-0.283243}{} &Zn &  \SI{}{} \\
			& \SI{24.9643+-0.334121}{} &Zn &  \SI{}{} \\ \hline
			
			15 & \SI{7.99436+-0.251832}{} &Zn &  \SI{}{} \\
			& \SI{8.88485+-0.218074}{} &Zn &  \SI{}{} \\ \hline
			
			16 (1-Cent)& \SI{7.98818+-0.234997}{} &Zn &  \SI{}{} \\
			& \SI{8.8637+-0.214616}{} &Zn &  \SI{}{} \\ \hline
			
			17 & \SI{7.98377+-0.225856}{} &Zn &  \SI{}{} \\
			& \SI{8.85058+-0.213236}{} &Zn &  \SI{}{} \\ \hline
			
			18 & \SI{7.41993+-0.228314}{} &Zn &  \SI{}{} \\
			& \SI{8.21747+-0.205213}{} &Zn &  \SI{}{} \\ \hline
			
			19 & \SI{5.40493+-0.248856}{} &Zn &  \SI{}{} \\
			& \SI{6.33056+-0.227739}{} &Zn &  \SI{}{} \\
			& \SI{7.05903+-0.435629}{} &Zn &  \SI{}{} \\ \hline
			
			20 (Kronkorken) & \SI{4.52319+-0.305646}{} &Zn &  \SI{}{} \\
			& \SI{6.35883+-0.22392}{} &Zn &  \SI{}{} \\
			& \SI{7.03423+-0.206604}{} &Zn &  \SI{}{} \\ \hline
			
			21 (Ag) & \SI{3.05354+-0.315386}{} &Zn &  \SI{}{} \\
			& \SI{22.0666+-0.247465}{} &Zn &  \SI{}{} \\
			& \SI{24.9605+-0.264321}{} &Zn &  \SI{}{} \\ \hline
		\end{tabular}
		\caption{Gemessene Röntgenfluoreszenzmaxima. Die Vergleichsenergien wurden dem Periodensystem des Programms Phywe Measure 4 entnommen.}
		\label{tb_peaks_unknown} 
	\end{table}
	
	\subsection{Diskussion}
	%TODO Bezug/Nutzen oder sonst was
	%TODO auch hier die Hypothese wiederholen
	%TODO keine Messwerte hier, nach manchen Menschen, zumindest "direkt" erstellte Diagramme net hier, auch wenn Lesbarkeit-bla
	
	\section{Schlussfolgerung}
	%TODO Rückgriff auf Hypothese und drittes Nennen dieser
	
	%TODO Quellen zitieren, Websiten mit Zugriffsdatum
	%TODO Verweise auf das Laborbuch (sind erlaubt)
	%TODO Tabelle + Bilder mit Beschriftung
	%\printbibliography
\end{document}
