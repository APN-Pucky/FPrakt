% Autor: Leonhard Segger, Alexander Neuwirth
% Datum: 2017-10-30
\documentclass[
	% Papierformat
	a4paper,
	% Schriftgröße (beliebige Größen mit „fontsize=Xpt“)
	12pt,
	% Schreibt die Papiergröße korrekt ins Ausgabedokument
	pagesize,
	% Sprache für z.B. Babel
	ngerman
]{scrartcl}

% Achtung: Die Reihenfolge der Pakete kann (leider) wichtig sein!
% Insbesondere sollten (so wie hier) babel, fontenc und inputenc (in dieser
% Reihenfolge) als Erstes und hyperref und cleveref (Reihenfolge auch hier
% beachten) als Letztes geladen werden!

\usepackage{tikz}
\usetikzlibrary{calc,patterns,angles,quotes} % loads some tikz extensions\usepackage{tikz}
\usetikzlibrary{babel}

% Silbentrennung etc.; Sprache wird durch Option bei \documentclass festgelegt
\usepackage{babel}
% Verwendung der Zeichentabelle T1 (Sonderzeichen etc.)
\usepackage[T1]{fontenc}
% Legt die Zeichenkodierung der Eingabedatei fest, z.B. UTF-8
\usepackage[utf8]{inputenc}
% Schriftart
\usepackage{lmodern}
% Zusätzliche Sonderzeichen
\usepackage{textcomp}

% Mathepaket (intlimits: Grenzen über/unter Integralzeichen)
\usepackage[intlimits]{amsmath}
% Ermöglicht die Nutzung von \SI{Zahl}{Einheit} u.a.
\usepackage{siunitx}
% Zum flexiblen Einbinden von Grafiken (\includegraphics)
\usepackage{graphicx}
% Abbildungen im Fließtext
\usepackage{wrapfig}
% Abbildungen nebeneinander (subfigure, subtable)
\usepackage{subcaption}
% Funktionen für Anführungszeichen
\usepackage{csquotes}
\MakeOuterQuote{"}
% Zitieren, Bibliografie
\usepackage[sorting=none]{biblatex}


% Zur Darstellung von Webadressen
\usepackage{url}
%chemische Formeln
\usepackage[version=4]{mhchem}
% siunitx: Deutsche Ausgabe, Messfehler getrennt mit ± ausgeben
\usepackage{floatrow}
\floatsetup[table]{capposition=top}
\usepackage{float}
% Verlinkt Textstellen im PDF-Dokument
\usepackage[unicode]{hyperref}
% "Schlaue" Referenzen (nach hyperref laden!)
\usepackage{cleveref}
\sisetup{
	locale=DE,
	separate-uncertainty
}
\bibliography{BA-C-04_MP1_21-01-2019_References}

\begin{document}

	\begin{titlepage}
		\centering
		{\scshape\LARGE Versuchsbericht zu \par}
		\vspace{1cm}
		{\scshape\huge MP1 - Eigenschaften der Solarzelle \par} %TODO Iwie ist der Titel nur 50%
		\vspace{2.5cm}
		{\LARGE Gruppe BA-C-04 \par}
		\vspace{0.5cm}

		{\large Alexander Neuwirth (E-Mail: a\_neuw01@wwu.de) \par}
		{\large Leonhard Segger (E-Mail: l\_segg03@uni-muenster.de) \par}
		\vfill

		durchgeführt am 21.01.2019\par
		betreut von\par
		{\large Finn Kutschmann}

		\vfill

		{\large \today\par}
	\end{titlepage}
	\tableofcontents
	\newpage

	%TODO mehr TODO in Default

	\section{Kurzfassung}
	% Hypothese	und deren Ergebnis, wenn Hypothese ist, dass nur Theorie erfüllt, sagen: Erwartung: Theorie aus einführung (mit reflink) erfüllt
	% Ergebnisse, auch Zahlen, mindestens wenn's halbwegs Sinn ergibt
	% Was wurde gemacht
	% manche leute wollen Passiv oder "man", manche nicht

  \section{Theorie}
	% wdh. Texte
	% wdh. Besprechung
	\subsection{Dotierstoffe}
	\subsection{Solarzelle}
	\begin{equation}
			\label{eq_photostrom}
			I = I_0 \left[\exp{\left(\frac{eU}{n k_B T}\right)}-1\right]-I_{ph}
	\end{equation}

	\section{Methoden}
	% Bilder von der Website klauen
	% einer will Präsens

	\section{Ergebnisse und Diskussion}

	\subsection{Dotierstoffe}
	\subsubsection{Unsicherheiten}
	\subsubsection{Beobachtung und Datenanalyse}

	\subsubsection{Diskussion}
	\subsection{Solarzelle}
	\subsubsection{Unsicherheiten}
	\subsubsection{Beobachtung und Datenanalyse}
	In \cref{fig_poly_unbeleuchtet_20} und \cref{fig_tandem_unbeleuchtet_20} sind die aufgenohmenen I-U-Kennlinien der polykistallinen Silizium Solarzelle und der Tandemzelle abgebildet.

	\begin{figure}[H]
			\includegraphics[width=.9\linewidth]{img/PolykristallineZelle_unbeleuchtet_20.pdf}
			\caption{
				I-U-Kennlinie einer polykristallinen Silizium Solarzelle bei Raumtemperatur.
								}
			\label{fig_poly_unbeleuchtet_20}
	\end{figure}

	\begin{figure}[H]
			\includegraphics[width=.9\linewidth]{img/Tandemzelle_unbeleuchtet_20.pdf}
			\caption{
				I-U-Kennlinie einer Tandemzelle bei Raumtemperatur.
								}
			\label{fig_tandem_unbeleuchtet_20}
	\end{figure}

	Die Fitfunktion ist die ideale Diodenkennlinie unter Einstrahlung von Licht \cref{eq_photostrom}.

	\cref{fig_tandem_beleuchtet_20} stellt exemplarisch die I-U-Kennlinie der Solarzellen dar, wenn sie mit Licht bestrahlt werden (weitere befinden sich im \nameref{s_anhang})
	Das rote (grüne) Rechteck gibt die im 4. Quadranten durch $U_{MPP}$ und $I_{MPP}$ ($U_{oc}$ und $I_{sc}$) begrenzte Fläche an.
	Die rote Fläche entspricht der Maximalen Leistung.
	Der Füllfaktor $FF$ ergibt sich als das Verhältnis der Flächen, also
	\begin{equation}
		FF = \frac{U_{mpp}I_{mpp}}{U_{oc}I_{sc}}.
	\end{equation}


	\begin{figure}[H]
			\includegraphics[width=.9\linewidth]{img/Tandemzelle_beleuchtet_20.pdf}
			\caption{
				I-U-Kennlinie einer Tandemzelle bei $T=\SI{20}{\degree}$ unter Beleuchtung.
								}
			\label{fig_tandem_beleuchtet_20}
	\end{figure}
	%TODO table


	\subsubsection{Diskussion}
	%TODO Messfehler bei unbeleuchtet poly???? diskus
	%TODO hab das hier hin geschpoben weiel diskus
	Da in \cref{eq_photostrom} $I_{ph}$ der vertikale Verschiebung an der Ordinate entspricht wird erwartet, dass dieser nahe Null ist, wenn die Solarzelle nicht beleuchtet wird, also in \cref{fig_poly_unbeleuchtet_20} und \cref{fig_tandem_unbeleuchtet_20}.
	%TODO passt bis auf thermo + hintergrund licht effekte


	% Allgemeine Beobachtungen
	% Einflüsse von veränderten Parametern auf Messung
	% Berechung nach Aufgabenstellung

	% Bezug/Nutzen oder sonst was
	% auch hier die Hypothese wiederholen
	% keine Messwerte hier, nach manchen Menschen, zumindest "direkt" erstellte Diagramme net hier, auch wenn Lesbarkeit-bla

	\section{Schlussfolgerung}
	% Rückgriff auf Hypothese und drittes Nennen dieser

	% Quellen zitieren, Websiten mit Zugriffsdatum
	% Verweise auf das Laborbuch (sind erlaubt)
	% Tabelle + Bilder mit Beschriftung
	%\printbibliography

	%TODO Anhang notwendig?
	\section{Anhang} \label{s_anhang}
	\subsection{Solarzelle}
	\begin{figure}[H]
			\includegraphics[width=.9\linewidth]{img/PolykristallineZelle_beleuchtet_20.pdf}
			\caption{
				I-U-Kennlinie einer polykristallinen Silizium Solarzelle bei Raumtemperatur unter Beleuchtung.
								}
			\label{fig_poly_beleuchtet_20}
	\end{figure}
	\begin{figure}[H]
			\includegraphics[width=.9\linewidth]{img/PolykristallineZelle_beleuchtet_35.pdf}
			\caption{
				I-U-Kennlinie einer polykristallinen Silizium Solarzelle bei $T=\SI{35}{\degree}$ unter Beleuchtung.
								}
			\label{fig_poly_beleuchtet_35}
	\end{figure}

	\begin{figure}[H]
			\includegraphics[width=.9\linewidth]{img/PolykristallineZelle_beleuchtet_50.pdf}
			\caption{
				I-U-Kennlinie einer polykristallinen Silizium Solarzelle bei $T=\SI{50}{\degree}$ unter Beleuchtung.
								}
			\label{fig_poly_beleuchtet_50}
	\end{figure}


	\begin{figure}[H]
			\includegraphics[width=.9\linewidth]{img/Tandemzelle_beleuchtet_50.pdf}
			\caption{
				I-U-Kennlinie einer Tandemzelle bei $T=\SI{50}{\degree}$ unter Beleuchtung.
								}
			\label{fig_tandem_beleuchtet_50}
	\end{figure}

	\begin{figure}[H]
			\includegraphics[width=.9\linewidth]{img/Tandemzelle_realbeleuchtet_20.pdf}
			\caption{
				I-U-Kennlinie einer Tandemzelle bei $T=\SI{20}{\degree}$ unter Beleuchtung durch einen AM-Filter.
				%TODO die Daten sagen nur hier "realbelecuhtet" also gehe ich von nur 1x AM-Filter aus
			}
			\label{fig_tandem_realbeleuchtet_20}
	\end{figure}
\end{document}
