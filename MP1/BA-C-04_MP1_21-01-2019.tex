% Autor: Leonhard Segger, Alexander Neuwirth
% Datum: 2017-10-30
\documentclass[
	% Papierformat
	a4paper,
	% Schriftgröße (beliebige Größen mit „fontsize=Xpt“)
	12pt,
	% Schreibt die Papiergröße korrekt ins Ausgabedokument
	pagesize,
	% Sprache für z.B. Babel
	ngerman
]{scrartcl}

% Achtung: Die Reihenfolge der Pakete kann (leider) wichtig sein!
% Insbesondere sollten (so wie hier) babel, fontenc und inputenc (in dieser
% Reihenfolge) als Erstes und hyperref und cleveref (Reihenfolge auch hier
% beachten) als Letztes geladen werden!

\usepackage{tikz}
\usetikzlibrary{calc,patterns,angles,quotes} % loads some tikz extensions\usepackage{tikz}
\usetikzlibrary{babel}

% Silbentrennung etc.; Sprache wird durch Option bei \documentclass festgelegt
\usepackage{babel}
% Verwendung der Zeichentabelle T1 (Sonderzeichen etc.)
\usepackage[T1]{fontenc}
% Legt die Zeichenkodierung der Eingabedatei fest, z.B. UTF-8
\usepackage[utf8]{inputenc}
% Schriftart
\usepackage{lmodern}
% Zusätzliche Sonderzeichen
\usepackage{textcomp}

% Mathepaket (intlimits: Grenzen über/unter Integralzeichen)
\usepackage[intlimits]{amsmath}
% Ermöglicht die Nutzung von \SI{Zahl}{Einheit} u.a.
\usepackage{siunitx}
% Zum flexiblen Einbinden von Grafiken (\includegraphics)
\usepackage{graphicx}
% Abbildungen im Fließtext
\usepackage{wrapfig}
% Abbildungen nebeneinander (subfigure, subtable)
\usepackage{subcaption}
% Funktionen für Anführungszeichen
\usepackage{csquotes}
\MakeOuterQuote{"}
% Zitieren, Bibliografie
\usepackage[sorting=none]{biblatex}


% Zur Darstellung von Webadressen
\usepackage{url}
%chemische Formeln
\usepackage[version=4]{mhchem}
% siunitx: Deutsche Ausgabe, Messfehler getrennt mit ± ausgeben
\usepackage{floatrow}
\floatsetup[table]{capposition=top}
\usepackage{float}
% Verlinkt Textstellen im PDF-Dokument
\usepackage[unicode]{hyperref}
% "Schlaue" Referenzen (nach hyperref laden!)
\usepackage{cleveref}
\sisetup{
	locale=DE,
	separate-uncertainty
}
\bibliography{BA-C-04_MP1_21-01-2019_References}

\begin{document}

	\begin{titlepage}
		\centering
		{\scshape\LARGE Versuchsbericht zu \par}
		\vspace{1cm}
		{\scshape\huge MP1 - Eigenschaften der Solarzelle \par} %TODO Iwie ist der Titel nur 50%
		\vspace{2.5cm}
		{\LARGE Gruppe BA-C-04 \par}
		\vspace{0.5cm}

		{\large Alexander Neuwirth (E-Mail: a\_neuw01@wwu.de) \par}
		{\large Leonhard Segger (E-Mail: l\_segg03@uni-muenster.de) \par}
		\vfill

		durchgeführt am 21.01.2019\par
		betreut von\par
		{\large Finn Kutschmann}

		\vfill

		{\large \today\par}
	\end{titlepage}
	\tableofcontents
	\newpage

	%TODO mehr TODO in Default

	\section{Kurzfassung}
	% Hypothese	und deren Ergebnis, wenn Hypothese ist, dass nur Theorie erfüllt, sagen: Erwartung: Theorie aus einführung (mit reflink) erfüllt
	% Ergebnisse, auch Zahlen, mindestens wenn's halbwegs Sinn ergibt
	% Was wurde gemacht
	% manche leute wollen Passiv oder "man", manche nicht



  \section{Theorie}
	% wdh. Texte
	% wdh. Besprechung


	%TODO siehe Methoden gegen dopplung

	\subsection{Halbleiter}
	%TODO dotierkram allgemein

	\subsection{Solarzelle}
	%TODO ultra ätzende Herleitung von pn-Übergang zu Solarzelle
	%TODO MPP erklären
	%TODO AM erklären
	\begin{equation}
			\label{eq_photostrom}
			I = I_0 \left[\exp{\left(\frac{eU}{n k_B T}\right)}-1\right]-I_{ph}
	\end{equation}

	\subsubsection{Vierspitzenmethode}
	Die Vierspitzenmethode ermöglicht die Messung des Flächenwiderstands in einer dünnen Probe.
	Dazu werden vier Spitzen, dies sich unter gleichem Abstand in einer Linie befinden, auf die Probenoberfläche gedrückt.
	Über die äußeren beiden lässt das Gerät dann einen bekannten Strom fließen und misst die Potenzialdifferenz zwischen den mittleren beiden.
	Die Vierspitzenmethode basiert auf dem Prinzip der Vierleitermessung nach Thomson und Kelvin, die es erlaubt den untersuchten Widerstand unabhängig vom Anschlusswiderstand (hier dem Übergangswiderstand Spitze zu Probe) zu bestimmen.
	%TODO dein Kram:
	\iffalse
	-allgemeine Formel
	-sehr dünne schichten Korrekturfaktor 1
	-hier: ...kp
	\fi

	\section{Methoden}
	% Bilder von der Website klauen
	% einer will Präsens
	\subsection{Solarzellen}
	Da der vorliegende Versuchsaufbau fehlerhaft war und es nicht möglich war die Messungen durchzuführen, wird hier nur das grundsätzliche Vorgehen beschrieben.

	Es sollen für eine Tandemsolarzelle und eine Solarzelle aus polykistallinem Silizium I-U-Kennlinien aufgenommen werden.
	Dies wird ohne Beleuchtung der Solarzellen, bei vollständigem Sonnenspektrum und bei einem AM-gefilterten Sonnenspektrum bei verschiedenen Temperaturen durchgeführt.
	Dazu wird an die Zellen eine Spannung angelegt, die schrittweise erhöht wird.
	Bei Beleuchtung hätte auch der Lastwiderstand variiert werden können, um die I-U-Kennlinie aufzunehmen.
	Dies hätte jedoch das Messen negativer Spannungen unmöglich gemacht, die so in die Fit-Funktion mit einbezogen werden konnten.

	\subsection{Dotierstoffe}
	Es wird ein Gerät verwendet, dass die Vierspitzenmethode verwendet, um den Widerstand der untersuchten Halbleiter zu bestimmen.



	\section{Ergebnisse und Diskussion}

	\subsection{Dotierstoffe}
	\subsubsection{Unsicherheiten}
	\subsubsection{Beobachtung und Datenanalyse}
	Mit der Vierspitzenmethode wurde der Widerstand $R=U/I$ gemessen.
	Daraus lässt sich dann mittels der Scheibendicke $d$ für dünne Scheiben ($d/s<0.5$, $s=\SI{1.59}{mm}$ Messspitzenabstand):
	\begin{equation}
			\rho = \frac{R d \pi}{\ln 2}
	\end{equation}
	\begin{equation}
			u(\rho) = \rho\cdot\sqrt{(u(R)/R)^2 + (u(d)/d)^2}
	\end{equation}
	Für die dickere Scheiben erhält man einen Korrekturfaktor gemäß DIN-NORM. %TODO cite DIN!
	Dieser beträgt für Probe 4 $0.84$ und für Probe 5 $0.83$.

	In \cref{tb_spez_wd} befinden sich die über mehrere Messungen gemittelten Messergebnisse.

	\begin{table}[H]
		\centering
		\begin{tabular}{c | c | c | c  }
			 &Widerstand $R$ in \si{\ohm}& Scheibendicke $d$ in \si{mm} & spez. Widerstand $\rho$ in \si{m\ohm m} \\ \hline
			 Probe 1, p-dot& \SI{50+-4}{}&\SI{0.4486+-0.0009}{}& \SI{101+-7}{} \\
			 Probe 2, n-dot&\SI{8.52+-0.21}{}&\SI{0.4532+-0.0009}{}&\SI{17.5+-0.4}{} \\
			 Probe 3, n-dot&\SI{0.0713+-0.0018}{}&\SI{0.0677+-0.001}{}&\SI{0.219+-0.005}{} \\
			 Probe 4, p-dot&\SI{0.00192+-0.00004}{}&\SI{1.999+-0.005}{}&\SI{0.0146+-0.0003}{} \\
			 Probe 6, n-dot&\SI{0.084+-0.007}{}&\SI{0.6826+-0.0009}{}&\SI{0.260+-0.022}{} \\
			 Probe 5, i-dot&\SI{25+-5}{}&\SI{2.1070+-0.0013}{}&\SI{199.2+-33.2}{}  \\ %TODO i-dot fine?
		\end{tabular}
		\caption{
		Silizium
		}
		\label{tb_spez_wd}
\end{table}
	\subsubsection{Diskussion}
	\subsection{Solarzelle}
	\subsubsection{Unsicherheiten}
	\subsubsection{Beobachtung und Datenanalyse}
	Da keine eigenen Messungen zu den Solarzellen durchgeführt werden konnten, werden im folgenden alte Messdaten, die uns freundlicherweise von Finn Kutschmann ausgehändigt wurden, ausgewertet. %TODO fine?

	In \cref{fig_poly_unbeleuchtet_20} und \cref{fig_tandem_unbeleuchtet_20} sind die aufgenohmenen I-U-Kennlinien der polykistallinen Silizium Solarzelle und der Tandemzelle abgebildet.

	\begin{figure}[H]
			\includegraphics[width=.9\linewidth]{img/PolykristallineZelle_unbeleuchtet_20.pdf}
			\caption{
				I-U-Kennlinie einer polykristallinen Silizium Solarzelle bei Raumtemperatur.
								}
			\label{fig_poly_unbeleuchtet_20}
	\end{figure}

	\begin{figure}[H]
			\includegraphics[width=.9\linewidth]{img/Tandemzelle_unbeleuchtet_20.pdf}
			\caption{
				I-U-Kennlinie einer Tandemzelle bei Raumtemperatur.
								}
			\label{fig_tandem_unbeleuchtet_20}
	\end{figure}

	Die Fitfunktion ist die ideale Diodenkennlinie unter Einstrahlung von Licht \cref{eq_photostrom}.

	\cref{fig_tandem_beleuchtet_20} stellt exemplarisch die I-U-Kennlinie der Solarzellen dar, wenn sie mit Licht bestrahlt werden (weitere befinden sich im \nameref{s_anhang}).
	Das rote (grüne) Rechteck gibt die im 4. Quadranten durch $U_{MPP}$ und $I_{MPP}$ ($U_{oc}$ und $I_{sc}$) begrenzte Fläche an.
	Die rote Fläche entspricht der Maximalen Leistung $P_{max} = U_{MPP}I_{MPP}$.
	Der Füllfaktor $FF$ ergibt sich als das Verhältnis der Flächen, also
	\begin{equation}
		FF = \frac{U_{mpp}I_{mpp}}{U_{oc}I_{sc}}.
	\end{equation}
	Dabei konnten $U_{MPP}$, $I_{MPP}$ und $I_{sc}$ direkt aus den Messpunkten extrahiert werden.
	$U_{oc}$ musste mittels des Fits interpoliert werden.


	\begin{figure}[H]
			\includegraphics[width=.9\linewidth]{img/Tandemzelle_beleuchtet_20.pdf}
			\caption{
				I-U-Kennlinie einer Tandemzelle bei $T=\SI{20}{\celsius}$ unter Beleuchtung.
								}
			\label{fig_tandem_beleuchtet_20}
	\end{figure}
	%TODO table
	In \cref{tb_solar_param_poly} und \cref{tb_solar_param_tandem} sind die wichtigsten Parameter der Solarzellen die unter Variation der Temperatur bestimmt wurden aufgeführt.

\begin{table}[H]
	%TODO signums? alles positiv??
		\centering
		\begin{tabular}{c | c | c | c  }
			 &$T=\SI{20}{\celsius}$& $T=\SI{35}{\celsius}$& $T=\SI{50}{\celsius}$ \\ \hline
			 $I_{sc}$ in \si{mA}& \SI{151+-0.3}{}&\SI{149+-0.3}{}& \SI{147+-0.3}{} \\
			 $U_{oc}$ in \si{V}&\SI{0.63+-0.04}{}&\SI{0.61+-0.04}{}&\SI{0.61+-0.05}{} \\
			 $I_{MPP}$ in \si{mA}&\SI{108+-0.3}{}&\SI{99+-0.3}{}&\SI{96+-0.3}{} \\
			 $U_{MPP}$ in \si{V}&\SI{0.31+-0.003}{}&\SI{0.32+-0.003}{}&\SI{0.31+-0.003}{} \\
			 $P_{MPP}$ in \si{mW}&\SI{33.48+-0.32}{}&\SI{31.68+-0.3}{}&\SI{29.76+-0.3}{} \\
			 $FF$ in \si{\percent}&\SI{35.05+-2.5}{}&\SI{34.8+-2.5}{}&\SI{33+-2.7}{} \\
		\end{tabular}
		\caption{
		Polykristalline Zelle.
		}
		\label{tb_solar_param_poly}
\end{table}

\begin{table}[H]
	%TODO signums? alles positiv??
		\centering
		\begin{tabular}{c | c | c | c  }
			 &$T=\SI{20}{\celsius}$& $T=\SI{50}{\celsius}$& AM-Filter $T=\SI{20}{\celsius}$ \\ \hline
			 $I_{sc}$ in \si{mA}& \SI{7.9+-0.003}{}&\SI{5.42+-0.003}{}& \SI{0.94+-0.003}{} \\
			 $U_{oc}$ in \si{V}&\SI{0.88+-0.04}{}&\SI{0.776+-0.032}{}&\SI{0.705+-0.024}{} \\
			 $I_{MPP}$ in \si{mA}&\SI{6.84+-0.03}{}&\SI{4.24+-0.003}{}&\SI{0.700+-0.003}{} \\
			 $U_{MPP}$ in \si{V}&\SI{0.64+-0.003}{}&\SI{0.60+-0.003}{}&\SI{0.54+-0.003}{} \\
			 $P_{MPP}$ in \si{mW}&\SI{4.378+-0.022}{}&\SI{2.544+-0.012}{}&\SI{0.378+-0.003}{} \\
			 $FF$ in \si{\percent}&\SI{63.3+-2.8}{}&\SI{60.5+-2.5}{}&\SI{57+-2.0}{} \\
		\end{tabular}
		\caption{
		Tandemzelle.
		}
		\label{tb_solar_param_tandem}
\end{table}
	\subsubsection{Diskussion}
	%TODO Messfehler bei unbeleuchtet poly???? diskus
	%TODO hab das hier hin geschpoben weiel diskus
	Da in \cref{eq_photostrom} $I_{ph}$ der vertikale Verschiebung an der Ordinate entspricht, wird erwartet, dass dieser nahe Null ist, wenn die Solarzelle nicht beleuchtet wird, also in \cref{fig_poly_unbeleuchtet_20} und \cref{fig_tandem_unbeleuchtet_20}.
	%TODO passt bis auf thermo + hintergrund licht effekte


	% Allgemeine Beobachtungen
	% Einflüsse von veränderten Parametern auf Messung
	% Berechung nach Aufgabenstellung

	% Bezug/Nutzen oder sonst was
	% auch hier die Hypothese wiederholen
	% keine Messwerte hier, nach manchen Menschen, zumindest "direkt" erstellte Diagramme net hier, auch wenn Lesbarkeit-bla

	\section{Schlussfolgerung}
	% Rückgriff auf Hypothese und drittes Nennen dieser

	% Quellen zitieren, Websiten mit Zugriffsdatum
	% Verweise auf das Laborbuch (sind erlaubt)
	% Tabelle + Bilder mit Beschriftung
	%\printbibliography

	%TODO Anhang notwendig?
	\section{Anhang} \label{s_anhang}
	\subsection{Solarzelle}
	\begin{figure}[H]
			\includegraphics[width=.9\linewidth]{img/PolykristallineZelle_beleuchtet_20.pdf}
			\caption{
				I-U-Kennlinie einer polykristallinen Silizium Solarzelle bei Raumtemperatur unter Beleuchtung.
								}
			\label{fig_poly_beleuchtet_20}
	\end{figure}
	\begin{figure}[H]
			\includegraphics[width=.9\linewidth]{img/PolykristallineZelle_beleuchtet_35.pdf}
			\caption{
				I-U-Kennlinie einer polykristallinen Silizium Solarzelle bei $T=\SI{35}{\celsius}$ unter Beleuchtung.
								}
			\label{fig_poly_beleuchtet_35}
	\end{figure}

	\begin{figure}[H]
			\includegraphics[width=.9\linewidth]{img/PolykristallineZelle_beleuchtet_50.pdf}
			\caption{
				I-U-Kennlinie einer polykristallinen Silizium Solarzelle bei $T=\SI{50}{\celsius}$ unter Beleuchtung.
								}
			\label{fig_poly_beleuchtet_50}
	\end{figure}


	\begin{figure}[H]
			\includegraphics[width=.9\linewidth]{img/Tandemzelle_beleuchtet_50.pdf}
			\caption{
				I-U-Kennlinie einer Tandemzelle bei $T=\SI{50}{\celsius}$ unter Beleuchtung.
								}
			\label{fig_tandem_beleuchtet_50}
	\end{figure}

	\begin{figure}[H]
			\includegraphics[width=.9\linewidth]{img/Tandemzelle_realbeleuchtet_20.pdf}
			\caption{
				I-U-Kennlinie einer Tandemzelle bei Raumtemperatur unter Beleuchtung durch einen AM-Filter.
				%TODO die Daten sagen nur hier "realbelecuhtet" also gehe ich von nur 1x AM-Filter aus
			}
			\label{fig_tandem_realbeleuchtet_20}
	\end{figure}
\end{document}
