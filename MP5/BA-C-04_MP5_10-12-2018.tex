% Autor: Leonhard Segger, Alexander Neuwirth
% Datum: 2017-10-30
\documentclass[
	% Papierformat
	a4paper,
	% Schriftgröße (beliebige Größen mit „fontsize=Xpt“)
	12pt,
	% Schreibt die Papiergröße korrekt ins Ausgabedokument
	pagesize,
	% Sprache für z.B. Babel
	ngerman
]{scrartcl}

% Achtung: Die Reihenfolge der Pakete kann (leider) wichtig sein!
% Insbesondere sollten (so wie hier) babel, fontenc und inputenc (in dieser
% Reihenfolge) als Erstes und hyperref und cleveref (Reihenfolge auch hier
% beachten) als Letztes geladen werden!

\usepackage{tikz}
\usetikzlibrary{calc,patterns,angles,quotes} % loads some tikz extensions\usepackage{tikz}
\usetikzlibrary{babel}

% Silbentrennung etc.; Sprache wird durch Option bei \documentclass festgelegt
\usepackage{babel}
% Verwendung der Zeichentabelle T1 (Sonderzeichen etc.)
\usepackage[T1]{fontenc}
% Legt die Zeichenkodierung der Eingabedatei fest, z.B. UTF-8
\usepackage[utf8]{inputenc}
% Schriftart
\usepackage{lmodern}
% Zusätzliche Sonderzeichen
\usepackage{textcomp}

% Mathepaket (intlimits: Grenzen über/unter Integralzeichen)
\usepackage[intlimits]{amsmath}
% Ermöglicht die Nutzung von \SI{Zahl}{Einheit} u.a.
\usepackage{siunitx}
% Zum flexiblen Einbinden von Grafiken (\includegraphics)
\usepackage{graphicx}
% Abbildungen im Fließtext
\usepackage{wrapfig}
% Abbildungen nebeneinander (subfigure, subtable)
\usepackage{subcaption}
% Funktionen für Anführungszeichen
\usepackage{csquotes}
\MakeOuterQuote{"}
% Zitieren, Bibliografie
\usepackage[sorting=none]{biblatex}


% Zur Darstellung von Webadressen
\usepackage{url}
%chemische Formeln
\usepackage[version=4]{mhchem}
% siunitx: Deutsche Ausgabe, Messfehler getrennt mit ± ausgeben
\usepackage{floatrow}
\floatsetup[table]{capposition=top}
\usepackage{float}
% Verlinkt Textstellen im PDF-Dokument
\usepackage[unicode]{hyperref}
% "Schlaue" Referenzen (nach hyperref laden!)
\usepackage{cleveref}
\sisetup{
	locale=DE,
	separate-uncertainty
}
\bibliography{BA-C-04_MP5_10-12-2018_References}

\begin{document}

	\begin{titlepage}
		\centering
		{\scshape\LARGE Versuchsbericht zu \par}
		\vspace{1cm}
		{\scshape\huge MP5 - Weiche Materie: Physik der
flüssigen Kristallen \par}
		\vspace{2.5cm}
		{\LARGE Gruppe BA-C-04 \par}
		\vspace{0.5cm}

		{\large Alexander Neuwirth (E-Mail: a\_neuw01@wwu.de) \par}
		{\large Leonhard Segger (E-Mail: l\_segg03@uni-muenster.de) \par}
		\vfill

		durchgeführt am 10.12.2018\par
		betreut von\par
		{\large Aaron Rigoni}

		\vfill

		{\large \today\par}
	\end{titlepage}
	\tableofcontents
	\newpage


	\section{Kurzfassung}
	% Hypothese	und deren Ergebnis, wenn Hypothese ist, dass nur Theorie erfüllt, sagen: Erwartung: Theorie aus einführung (mit reflink) erfüllt
	% Ergebnisse, auch Zahlen, mindestens wenn's halbwegs Sinn ergibt
	% Was wurde gemacht
	% manche leute wollen Passiv oder "man", manche nicht

	\section{Theorie}
	% wdh. Texte
	% wdh. Besprechung
	\section{Methoden}
	% Bilder von der Website klauen
	% einer will Präsens
	%TODO I guess hier der LCD-Bau, weil sonst (bis auf Diskussion) macht es net so viel Sinn

	\section{Ergebnisse und Diskussion}
	\subsection{Bestimmung der Kennlinien der LC-Displays}
	\subsubsection{Unsicherheiten}
	Beide Spannungen werden von einer digitalen Anzeige abgelesen.
	Mit einer Rechteckverteilung ergibt sich für die Spannung $U_{LCD}$ mit welcher das Display betrieben wird $u(U_{LCD})=\SI{0.00003}{V}$ und für die der Photodiode $u(U_{Ph})=\SI{0.0003}{V}$.
	Da beide Anzeigen stark in den letzten Ziffern schwanken, wird die Breite der Schwankungen mit
	\begin{itemize}
		\item $u(U_{LCD})=\SI{0.001}{V}$
		\item $u(U_{Ph})=\SI{0.003}{V}$
	\end{itemize}
	abgeschätzt, wohingegen die Displayunsicherheiten verschwinden.
	Da die Schwankungen der Diodenspannung deutlich größer im Bereich von 10\% bis 90\% Transmission ist, wird hierfür die doppelte Unsicherheit verwendet.
	\subsubsection{Beobachtung und Datenanalyse}
	%TODO Allg Beob
	Um von der gemessenen Spannung an der Photodiode auf auf die relative Transmission zu schließen wird. %TODO erklären wie Diode Licht intensität messen tut?
	\begin{equation}
			T = \frac{U-U_{max}}{U_{max}-U_{min}}
	\end{equation}
	verwendet. Wobei sich die kombinierte Standardunsicherheit
	\begin{align*}
		u(T) &= \sqrt{\sum_{i}\left(\frac{\partial T}{\partial x_i} u(x_i)\right)^2}\\
		&= \sqrt{\left(\frac{u(U)}{U_{max}-U_{min}}\right)^2+\left(\frac{u(U_{max})(U_{min}-U)}{(U_{max}-U_{min})^2}\right)^2+\left(\frac{u(U_{min})(U-U_{max})}{(U_{max}-U_{min})^2}\right)^2} \\
		&=	\frac{\sqrt{ (\alpha^2+1) (U_{max}-U_{min})^2+2U_{max}U_{min}+2U(U-U_{max}-U_{min})}}{(U_{max}-U_{min})^2} u(U_{max}) \\
	\end{align*}
	mit
	\begin{equation*}
			u(U)=\alpha u(U_{max})=\alpha u(U_{min})
	\end{equation*}
	ergibt.
	In \cref{fig_industry} und \cref{fig_selfmade} sind die gemessenen Kennlinien abgebildet.
	\begin{figure}[H]
			\includegraphics[width=1\linewidth]{images/industry.pdf}
			\caption{Kennlinie des industriell hergestellten LC-Displays.
			Die schwarze Senkrechte beschreibt die Schwellspannung, welche mindestens am Display anliegen muss um eine Veränderung messen zu können.
			Die rote(grüne) bespreibt dagegen eine Transmission von 90\%(10\%) durch das Diplay.  %TODO meh deutsch
			}
			\label{fig_industry}
	\end{figure}
	\begin{figure}[H]
			\includegraphics[width=1\linewidth]{images/selfmade.pdf}
			\caption{Kennlinie des selbstgebauten LC-Displays.
			Die schwarze Senkrechte beschreibt die Schwellspannung, welche mindestens am Display anliegen muss um eine Veränderung messen zu können.
			Die rote(grüne) bespreibt dagegen eine Transmission von 90\%(10\%) durch das Diplay.  %TODO meh deutsch
			}
			\label{fig_selfmade}
	\end{figure}
	Aus den Kennlinien lassen sich die Kenngrößen aus \cref{tb_kenngroessen} bestimmen.
	\begin{table}[H]
		\centering
		\begin{tabular}{ c | c | c }
			 & industriell & selbstgebaut \\ \hline
			$\Delta U=U_{10}-U_{90}$&\SI{2.0009+-0.0016}{V}&\SI{1.8348+-0.0016}{V} \\
			$U_{th}$ & \SI{3.211+-0.0012}{V} & \SI{0.0016+-0.0012}{V} \\ \hline
		\end{tabular}
		\caption{Kenngrößen des selbstgebauten und industriell gefertigten Displays.}
		\label{tb_kenngroessen}
\end{table}

	\subsubsection{Diskussion}
	%TODO Schwellspannung Unterschied U_th Grund?Nutzen?
	%TODO U_th hängt von Anisotropie Delta epsilon ab (vgl. -> Einleitung)
	%TODO Delta U ist bei selfmade kleiner => gut für Kontrastreiche Hell-Dunkel Displays, aber! selfmade beim ist die absolute Diodenspannung immernoch 10x heller. (soll des iwo in  Beobachtung?, passt jedenfalls zu Augen-beobachtung)
	\subsection{Bestimmung der kritischen Temperatur eines Flüssigkristalls}
	\subsubsection{Unsicherheiten}
	\subsubsection{Beobachtung und Datenanalyse}
	\subsubsection{Diskussion}

	% Allgemeine Beobachtungen
	% Einflüsse von veränderten Parametern auf Messung
	% Berechung nach Aufgabenstellung

	% Bezug/Nutzen oder sonst was
	% auch hier die Hypothese wiederholen
	% keine Messwerte hier, nach manchen Menschen, zumindest "direkt" erstellte Diagramme net hier, auch wenn Lesbarkeit-bla

	\section{Schlussfolgerung}
	% Rückgriff auf Hypothese und drittes Nennen dieser

	% Quellen zitieren, Websiten mit Zugriffsdatum
	% Verweise auf das Laborbuch (sind erlaubt)
	% Tabelle + Bilder mit Beschriftung
	%\printbibliography
\end{document}
