% Autor: Leonhard Segger, Alexander Neuwirth
% Datum: 2017-10-30
\documentclass[
	% Papierformat
	a4paper,
	% Schriftgröße (beliebige Größen mit „fontsize=Xpt“)
	12pt,
	% Schreibt die Papiergröße korrekt ins Ausgabedokument
	pagesize,
	% Sprache für z.B. Babel
	ngerman
]{scrartcl}

% Achtung: Die Reihenfolge der Pakete kann (leider) wichtig sein!
% Insbesondere sollten (so wie hier) babel, fontenc und inputenc (in dieser
% Reihenfolge) als Erstes und hyperref und cleveref (Reihenfolge auch hier
% beachten) als Letztes geladen werden!

\usepackage{tikz}
\usetikzlibrary{calc,patterns,angles,quotes} % loads some tikz extensions\usepackage{tikz}
\usetikzlibrary{babel}

% Silbentrennung etc.; Sprache wird durch Option bei \documentclass festgelegt
\usepackage{babel}
% Verwendung der Zeichentabelle T1 (Sonderzeichen etc.)
\usepackage[T1]{fontenc}
% Legt die Zeichenkodierung der Eingabedatei fest, z.B. UTF-8
\usepackage[utf8]{inputenc}
% Schriftart
\usepackage{lmodern}
% Zusätzliche Sonderzeichen
\usepackage{textcomp}

% Mathepaket (intlimits: Grenzen über/unter Integralzeichen)
\usepackage[intlimits]{amsmath}
% Ermöglicht die Nutzung von \SI{Zahl}{Einheit} u.a.
\usepackage{siunitx}
% Zum flexiblen Einbinden von Grafiken (\includegraphics)
\usepackage{graphicx}
% Abbildungen im Fließtext
\usepackage{wrapfig}
% Abbildungen nebeneinander (subfigure, subtable)
\usepackage{subcaption}
% Funktionen für Anführungszeichen
\usepackage{csquotes}
\MakeOuterQuote{"}
% Zitieren, Bibliografie
\usepackage[sorting=none]{biblatex}


% Zur Darstellung von Webadressen
\usepackage{url}
%chemische Formeln
\usepackage[version=4]{mhchem}
% siunitx: Deutsche Ausgabe, Messfehler getrennt mit ± ausgeben
\usepackage{floatrow}
\floatsetup[table]{capposition=top}
\usepackage{float}
% Verlinkt Textstellen im PDF-Dokument
\usepackage[unicode]{hyperref}
% "Schlaue" Referenzen (nach hyperref laden!)
\usepackage{cleveref}
\sisetup{
	locale=DE,
	separate-uncertainty
}
%\bibliography{BA-C-04_MP6_07-01-2019_References}

\begin{document}

	\begin{titlepage}
		\centering
		{\scshape\LARGE Versuchsbericht zu \par}
		\vspace{1cm}
		{\scshape\huge MP6 - Metallische Gläser \par}
		\vspace{2.5cm}
		{\LARGE Gruppe BA-C-04 \par}
		\vspace{0.5cm}

		{\large Alexander Neuwirth (E-Mail: a\_neuw01@wwu.de) \par}
		{\large Leonhard Segger (E-Mail: l\_segg03@uni-muenster.de) \par}
		\vfill

		durchgeführt am 07.01.2019\par
		betreut von\par
		{\large Manoel da Silva Pinto}

		\vfill

		{\large \today\par}
	\end{titlepage}
	\tableofcontents
	\newpage

	\section{Kurzfassung}
	% Hypothese	und deren Ergebnis, wenn Hypothese ist, dass nur Theorie erfüllt, sagen: Erwartung: Theorie aus einführung (mit reflink) erfüllt
	% Ergebnisse, auch Zahlen, mindestens wenn's halbwegs Sinn ergibt
	% Was wurde gemacht
	% manche leute wollen Passiv oder "man", manche nicht

  \section{Theorie}
	% wdh. Texte
	% wdh. Besprechung

	\section{Methoden} %TODO Ziele angeben
	% Bilder von der Website klauen
	% einer will Präsens
	Es sollen drei verschiedene Untersuchungen bezüglich des Verhaltens von metallischen Gläsern durchgeführt werden.
	Hierzu wird als Vertreter der metallischen Gläser PdNiP verwendet.

	\subsection{Kalorimetrie}
	Es wird ein Leistungskalorimeter verwendet, um den Wärmefluss beim Erhitzen von verschiedenen Stoffen zu untersuchen.
	Zunächst wird eine Bleiprobe, die sich bereits in einem Tiegel befindet, im Vergleich zu einem Referenztiegel untersucht.
	Bei dieser wird keine gläserne Phase erwartet, da dies ein Erhitzen über den Schmelzpunkt gefolgt von einem schnelleren Abkühlen, als es mit diesem oder einem anderen bekannten Aufbau möglich ist, erfordern würde.
	Es werden zwei aufeinanderfolgende Messzyklen durchgeführt und das auf der Probe angegebene Gewicht notiert.

	Dann werden von einem Band aus amorphem PdNiP drei ungefähr gleich große Stücke mit einer Zange abgekniffen, gewogen und in Tiegeln verschlossen.
	Diese werden im Vergleich zu einem leeren Referenztiegel im Kalorimeter bei je unterschiedlicher Heizrate untersucht.
	Auch hier werden zwei Messzyklen aufgenommen.

	\subsection{Röntgendiffraktometrie} %lul steht in der Anleitung mindestens einmal falsch geschrieben
	In einem Röntgendiffraktometer wird im Debye-Scherrer-Verfahren die Röngenbeugung in Abhängigkeit vom Einstrahlwinkel unter monochromatischer Röntgenstrahlung gemessen.
	Dabei wird eine amorphe und eine kristalline PdNiP-Probe untersucht.
	Beiden werden nicht pulverisiert.	%TODO weil pulver erwartung, dass bei kristall nichts exaktes

	\subsection{Messung der Vickershärte}
 	Es wird ein Mikroindenter verwendet, um die Vickershärte einer Probe aus kristallinem und einer aus amorphem PdNiP zu bestimmen.%TODO das steht in der Anleitung, aber an den Daten steht was von Paradin-Phosphor (was nicht existiert) und Ni.
	Dazu wird ein Diamant in Form einer vierseitigen Pyramide mit einem Öffnungswinkel von \SI{136}{\degree} für \SI{5}{s} mit einer Kraft von \SI{500}{\newton} auf die Probe gedrückt.
	Dann wird mit einem Mikroskop die Länge der beiden Diagonalen gemessen.
	Das verwendete Gerät errechnet dann aus diesen Werten die Vickershärte.
	Es werden jeweils \num{10} Messungen durchgeführt, um einen Mittelwert bilden zu können.

	\section{Ergebnisse und Diskussion}
	%TODO Unsicherheiten


	\subsection{Beobachtung und Datenanalyse}
	% Allgemeine Beobachtungen
	% Einflüsse von veränderten Parametern auf Messung
	\subsubsection{Unsicherheiten}
	% Berechung nach Aufgabenstellung

	\subsection{Diskussion}
	% Bezug/Nutzen oder sonst was
	% auch hier die Hypothese wiederholen
	% keine Messwerte hier, nach manchen Menschen, zumindest "direkt" erstellte Diagramme net hier, auch wenn Lesbarkeit-bla

	\section{Schlussfolgerung}
	% Rückgriff auf Hypothese und drittes Nennen dieser

	% Quellen zitieren, Websiten mit Zugriffsdatum
	% Verweise auf das Laborbuch (sind erlaubt)
	% Tabelle + Bilder mit Beschriftung
	%\printbibliography
\end{document}
