% Autor: Leonhard Segger, Alexander Neuwirth
% Datum: 2017-10-30
\documentclass[
	% Papierformat
	a4paper,
	% Schriftgröße (beliebige Größen mit „fontsize=Xpt“)
	12pt,
	% Schreibt die Papiergröße korrekt ins Ausgabedokument
	pagesize,
	% Sprache für z.B. Babel
	ngerman
]{scrartcl}

% Achtung: Die Reihenfolge der Pakete kann (leider) wichtig sein!
% Insbesondere sollten (so wie hier) babel, fontenc und inputenc (in dieser
% Reihenfolge) als Erstes und hyperref und cleveref (Reihenfolge auch hier
% beachten) als Letztes geladen werden!

\usepackage{tikz}
\usetikzlibrary{calc,patterns,angles,quotes} % loads some tikz extensions\usepackage{tikz}
\usetikzlibrary{babel}

% Silbentrennung etc.; Sprache wird durch Option bei \documentclass festgelegt
\usepackage{babel}
% Verwendung der Zeichentabelle T1 (Sonderzeichen etc.)
\usepackage[T1]{fontenc}
% Legt die Zeichenkodierung der Eingabedatei fest, z.B. UTF-8
\usepackage[utf8]{inputenc}
% Schriftart
\usepackage{lmodern}
% Zusätzliche Sonderzeichen
\usepackage{textcomp}

% Mathepaket (intlimits: Grenzen über/unter Integralzeichen)
\usepackage[intlimits]{amsmath}
% Ermöglicht die Nutzung von \SI{Zahl}{Einheit} u.a.
\usepackage{amssymb}
% mehr symbole plox
\usepackage{siunitx}
% Zum flexiblen Einbinden von Grafiken (\includegraphics)
\usepackage{graphicx}
% Abbildungen im Fließtext
\usepackage{wrapfig}
% Abbildungen nebeneinander (subfigure, subtable)
\usepackage{subcaption}
% Funktionen für Anführungszeichen
\usepackage{csquotes}
\MakeOuterQuote{"}
% Zitieren, Bibliografie
\usepackage[sorting=none]{biblatex}


% Zur Darstellung von Webadressen
\usepackage{url}
%chemische Formeln
\usepackage[version=4]{mhchem}
% siunitx: Deutsche Ausgabe, Messfehler getrennt mit ± ausgeben
\usepackage{floatrow}
\floatsetup[table]{capposition=top}
\usepackage{float}
% Verlinkt Textstellen im PDF-Dokument
\usepackage[unicode]{hyperref}
% "Schlaue" Referenzen (nach hyperref laden!)
\usepackage{cleveref}
\sisetup{
	locale=DE,
	separate-uncertainty
}
\bibliography{BA-C-04_SLM_27-05-2019_References}

\begin{document}

	\begin{titlepage}
		\centering
		{\scshape\LARGE Versuchsbericht zu \par}
		\vspace{1cm}
		{\scshape\huge Räumlicher LC-Modulator und diffraktive Optik \par}
		\vspace{2.5cm}
		{\LARGE Gruppe BA-C-04 \par}
		\vspace{0.5cm}

		{\large Alexander Neuwirth (E-Mail: a\_neuw01@wwu.de) \par}
		{\large Leonhard Segger (E-Mail: l\_segg03@uni-muenster.de) \par}
		\vfill

		durchgeführt am 27.05.2019\par
		betreut von\par
		{\large Milena Merkel}

		\vfill

		{\large \today\par}
	\end{titlepage}
	\tableofcontents
	\newpage

	\section{Kurzfassung}
	% Hypothese	und deren Ergebnis, wenn Hypothese ist, dass nur Theorie erfüllt, sagen: Erwartung: Theorie aus einführung (mit reflink) erfüllt
	% Ergebnisse, auch Zahlen, mindestens wenn's halbwegs Sinn ergibt
	% Was wurde gemacht
	% manche leute wollen Passiv oder "man", manche nicht

  \section{Theorie}
	% wdh. Texte
	% wdh. Besprechung

	\subsection{Gesetz von Malus}

	\subsection{Flüssigkristallzellen}
	%TODO ist ultra von dem anderen Versuch gecopypasted, aber das sollte ja wohl erlaubt sein.
	%TODO vlt mal in das alte Protokoll (MP5) schauen (kp, ob der das als pdf oder ausgedruckt wiedergegeben hat) und schauen, ob hier irgendwas falsch war.

	Flüssigkristalle zeichnen sich dadurch aus, dass sie mindestens unter bestimmten Bedingungen eine Orientierungsfernordnung, aber keine vollständige Positionsfernordnung, aufweisen, womit sie sich in letzterem Punkt von \enquote{gewöhnlichen} Kristallen unterscheiden.
	Positionsfernordnung bezeichnet hierbei eine Anordnung im Raum, die sich durch das periodische Fortsetzen einer Einheitszelle beschreiben lässt.
	Eine unvollständige Positionsfernordnung meint eine Periodizität in nur einer oder nur zwei Dimensionen.
	Orientierungsfernordnung meint, dass eine Vorzugsrichtung der Längsachsen der Moleküle des Flüssigkristalls existiert.
	Diese ist periodisch ortsabhängig oder konstant.
	Der Grad der Ordnung lässt sich als die Größe der durchschnittlichen Abweichung von der Vorzugsrichtung verstehen.

	Für die Herstellung von Flüssigkristallanzeigen wird ein Flüssigkristall in cholesterischer Phase verwendet. %TODO hier stand in der Anleitung immer "twisted nematic". Denke, das ist das gleiche.
	Dies meint, dass die Molekülachsen ebenenweise in die gleiche Richtung zeigen.
	Die Orientierung in den Ebenen ist dabei periodisch, da der Direktor (die Vorzugsachse) sich, wenn in Laufrichtung senkrecht zur Ebene betrachtet, mit konstanter Winkelgeschwindigkeit dreht und  eine Helixstruktur bildet, wenn das Ende des Direktorvektors im Raum verfolgt wird.

	\begin{figure}[H]
			\includegraphics[width=1\linewidth]{img/displayHelix}
			\caption{
			Polarisationsrichtung in der Flüssigkristallanzeige. a) ohne angelegte Spannung. b) bei angelegter Spannung. E1, E2: Elektroden (ITO); G: Glasplatten; I: Lichtintensität; L: Lichtwelle; LC: Flüssigkristall;
P1, P2: Polarisatoren; S: Schalter, V: Spannungsquelle. \cite{displayHelix}
			}
			\label{fig_displayHelixBild}
	\end{figure}

	Dies erlaubt den Bau einer Flüssigkristallanzeige, da in einem Flüssigkristall in cholesterischer Mesophase die Polarisation des Lichtes, wenn es entlang besagter Laufrichtung einfällt, der Helixform des Direktors folgt.
	Wenn bekannt ist, mit welcher Ganghöhe (Länge der Helix bei einer Umdrehung des Direktors) der Kristall vorliegt, kann auf der Ausgangsseite des Displays linear polarisiertes Licht einen Polfilter passieren, wenn dieser zur Einfallsrichtung soweit gedreht ist, wie sich der Direktor über die Dicke des Flüssigkristalls im Display dreht.
	Häufig wird hierfür eine Vierteldrehung verwendet. Wie gleich deutlich wird, sind hier nur Drehungen um $\frac{\pi (2z+1)}{2}$ mit $z \in \mathbb{Z}$ praktikabel.
	Da für die lineare Polarisation des einfallenden Lichts ebenfalls ein Polfilter verwendet wird, ist eine solche Flüssigkristallanzeige bidirektional.
	Wenn jetzt im Flüssigkristall ein annähernd homogenes elektrisches Feld angelegt wird, richten sich die Molekülachsen statt nach der Helixstruktur mit zunehmender Feldstärke zunehmend eher nach dem Feld aus (vgl. \cref{fig_displayHelixBild}). %TODO , wenn es sich bei diesem um Dipole handelt.
	Dies verhindert, dass sich die Polarisation des Lichtes im Innern ändert, weshalb bei einer der oben genannten Drehungen zwischen den Durchlassrichtungen der Polarisationsfilter das Licht den zweiten Polarisationsfilter nicht passieren kann.
	Jetzt wird auch klar, warum nur die oben genannten Drehungen praktikabel sind:
	Die Polarisationsfilter müssen senkrecht zueinander stehen, damit im Falle von angelegter Spannung die Transmission minimal wird (vgl. Gesetz von Malus).

	Demnach ist eine spannungsgesteuerte Schaltung der Durchlässigkeit des Filters möglich und wenn hinter das Display eine Lichtquelle oder ein Spiegel gebracht wird, kann die Zelle als Pixel einer größeren Anzeige verwendet werden.
	Bei Farbdisplays werden Pixel aus drei Subpixeln zusammengesetzt, welche wiederum aus einer spannungsgesteuerten Flüssigkristallanzeige mit einem Farbfilter (Rot, Grün, Blau) bestehen. %TODO das ist für den Versuch recht egal...

	Flüssigkristalldisplays werden mit Wechselspannung betrieben, da eine Gleichspannung die elektrolytische Zersetzung des Flüssigkristalls zufolge hätte. %TODO wurde jetzt auch nirgendwo angesprochen.

\subsection{Doppelbrechung}
%TODO sie meinte, dass dies für die Polarisationsänderung in der Helix verantwortlich wäre. Sehe ich nicht so recht. sollte man aber erwähnen, wenn es true ist. ergibt sich irgendwie in dem Abschnitt 1.1.3 wegen der Jones-Rechnung.

	Flüssigkristalle weisen einen polarisationsabhängigen Brechungsindex auf, da die Molekülelektronen je nach Orientierung der Moleküle und Schwingungsrichtung sich gegenseitig unterschiedlich stark zum Schwingen anregen.
	Dies führt bei einem Strahl, der nicht bereits entweder senkrecht oder parallel zur optischen Achse polarisiert ist, zum Phänomen der Doppelbrechung.
	Der Strahl spaltet in zwei senkrecht zueinander polarisierte Strahlen auf.
	Diese Aufspaltung wird maximal, wenn der einfallende Strahl senkrecht zum Direktor steht.
	%Wenn der Flüssigkristall in einem keilförmigen Gefäß ist, behält der Strahl nach dem Austritt aus dem Kristall eine Winkelaufspaltung, weshalb in größerem Abstand die Aufspaltung einfacher gemessen werden kann (vgl. \cref{fig_keil}).
	%In einem plan-parallelen Aufbau würden die Strahlen nach dem Austritt aus dem Kristall wieder parallel verlaufen und die Ortsaufspaltung würde nicht mit dem Messabstand steigen.
  %Die Differenz der Brechungsindizes ist dabei direkt abhängig vom Ordnungsgrad, weshalb sie als ein Maß für diesen verwendet werden kann.

	Bei steigender Temperatur steigt die mittlere Abweichung der Molekülachsen vom Direktor und damit sinkt der Ordnungsgrad und die Differenz der Brechungsindizes.
	Ab einer bestimmten, materialabhängigen kritischen Temperatur verliert der Flüssigkristall seine Orientierungsfernordnung vollständig und das Phänomen der Doppelbrechung verschwindet.
	%TODO deswegen sind alle falschen Ergebnisse auf mangelnde Kühlung zurückzuführen.
	%TODO brauch man das?


	\subsection{Diffraktive Optische Elemente} %alles groß, weil Überschrift?
	%TODO Jones-Vektoren, Jones-Formalismus?

	Mithilfe einer Matrix aus Flüssigkristallzellen lässt sich ein räumlicher Lichtmodulator realisieren.
	%TODO 1.1.7 TN-LC-Zelle + Polarisator ergibt Amplitudenmodulation (und auch Phase)
	Polarisator, Mikrodisplay und Analysator sorgen in dafür, dass ein Phasenunterschied zwischen den Flüssigkristallzellen entsteht, der proportional zum eingestellten Grauwert ist.
	Dies ergibt sich aus der Betrachtung des Systems mittels des Jones-Formalismus.
	Durch einzelne Ansteuerung der Flüssigkristallzellen entsteht ein räumliche Lichtmodulation.

	%TODO Interferenz?

	\subsection{Kohärenz}
	Der signifikante Unterschied zwischen einem Laser und einer Glimmlampe besteht in der Fähigkeit zur Interferenz.
	Die Lichtwellen, die den Laser verlassen haben eine feste Phasenbeziehung, weshalb nach Trennung des Strahls Interferenz zwischen beiden Teilstrahlen zu beobachten ist.

	Bei der qualitativen Beschreibung der Interferenzfähigkeit unterscheidet man zwischen räumlicher und zeitlicher Kohärenz.
	Zeitliche Kohärenz kann über Autokorrelation mittels des Michelson-Interferometers gemessen werden.
	Die Kohärenzlänge ist definiert durch den Abstand (optische Wegdifferenz), der die Länge eines Arms des Interferometers vom Interferenzmaximum wegbewegt werden kann, bevor die durch die Interferenz erzeugte Autokorrelationsfunktion nicht mehr über $1/\text{e}$ steigt.
	Die Kohärenzlänge ergibt sich aus der Monochromatizität des Lichts, da sich bei perfekt monochromatischem Licht die Phasenbeziehung über keine Entfernung ändert.
	Die räumliche Kohärenz kann mittels eines Doppelspaltexperiments (oder des Young-Interferometers) gemessen werden, da sie die Phasenbeziehung innerhalb einer Wellenfront beschreibt und der Doppelspalt Interfernz zwischen unterschiedlichen Teilen der Wellenfront realisiert.

	%TODO Fraunhofer-Beugung?
	%TODO Was erwarten wir bei Doppelspalt, gitter, einzelspalt, etc.
	%TODO Was sind DOEs?

	\section{Methoden}
	% Bilder von der Website klauen
	% einer will Präsens

	\subsection{Überprüfung des Gesetzes von Malus}
	Um das Gesetz von Malus zu überprüfen, wird das Licht eines He-Ne-Lasers durch einen Analysator geschickt und dann mit einer Linse auf ein Intensitätsmessgerät geschickt. %TODO He-Ne können wir nicht confirmen, aber ist schon ziemlich safe.
	Hierbei wird vermutet, dass der Laser bereits linear polarisiertes Licht erzeugt.
	Der Analysator wird in \SI{10}{\degree}-Schritten um \SI{360}{\degree} rotiert.

	\subsection{Einstellung der Eingangspolarisation}
	Der im Folgenden verwendete LC-Modulator weist einen von der Eingangspolarisation abhängigen Kontrast auf.
	Der maximale Kontrast soll bei einer Eingangspolarisation von \SI{-45}{\degree} bzw. \SI{-135}{\degree} liegen.
	Um diese Eingangspolarisation zu erreichen wird ein Analysator in den Strahlengang gebracht und auf \SI{-135}{\degree} eingestellt.
	Dann wird der Laser rotiert, bis die gemessene Intensität maximal ist.
	Sobald diese Position gefunden ist, wird zwischen Laser und Analysator der LC-Modulator eingebaut, der Strahl kollimiert und mit einer Kamera das sich ergebende Bild aufgenommen.
	Auf dem LC-Modulator wird ein horizontal geteiltes Bild eingestellt (eine Hälfte Weiß, eine Schwarz).
	Dann wird der Analysator um je \SI{90}{\degree} gedreht und jeweils ein Kamerabild aufgenommen, um zu zeigen, dass bei der gewählten Eingangspolarisation der Konstrast tatsächlich maximal ist. %TODO man sollte hier streng genommen bei einem Winkel von 90deg ein invertiertes Bild verwenden. not sure why, haben wir aber nicht gemacht.

	Zusätzlich wird der Kontrast gemessen, indem ein komplett weißes (schwarzes) Bild verwendet wird und mit einem Intensitätsmessgerät nach Fokussierung durch eine Linse die Intensität in der Fokusebene gemessen wird.

	\subsection{Bestimmung der Pixelgröße}
	Um die Pixelgröße des LC-Modulators zu bestimmen, wird ein weißes Bild mit einem 200x200 Pixel großen schwarzen Quadrat in der Mitte auf den Modulator gegeben.
	In den Strahlengang wird der LC-Modulator und im Abstand der Brennweite eine Linse dahinter gestellt, hinter der wieder im Abstand der Brennweite ein Schirm aufgestellt wird.
	Auf dem Schirm wird die Größe des Quadrats mit einem Lineal gemessen. %TODO waren wir hier wirklich nicht mit dem Modulator im Fokus? Hatten wir hier überhaupt einen Polarisator, an dem wir hätten falsch messen können?

	\subsection{Zusammenhang zwischen Grauwert und Polarisationszustand}
	

	\section{Ergebnisse und Diskussion}
	%TODO Unsicherheiten


	\subsection{Beobachtung und Datenanalyse}
	% Allgemeine Beobachtungen
	% Einflüsse von veränderten Parametern auf Messung
	\subsubsection{Unsicherheiten}
	% Berechung nach Aufgabenstellung

	\subsection{Diskussion}
	% Bezug/Nutzen oder sonst was
	% auch hier die Hypothese wiederholen
	% keine Messwerte hier, nach manchen Menschen, zumindest "direkt" erstellte Diagramme net hier, auch wenn Lesbarkeit-bla

	\section{Schlussfolgerung}
	% Rückgriff auf Hypothese und drittes Nennen dieser

	% Quellen zitieren, Websiten mit Zugriffsdatum
	% Verweise auf das Laborbuch (sind erlaubt)
	% Tabelle + Bilder mit Beschriftung
	%\printbibliography
\end{document}
